\chapter{Nazwa rozdziału}

\section{Rozdział}
Zażółć gęślą jaźń :). Polskie znaki działają podczas kopiowania pdfa czyli działają też w systemie anty-plagiatowym. Pierwszy wiersz automatycznie wstawia akapit. \\ %nowa linia
\indent Zobacz do kodu jeżeli chcesz wstawić dodatkowy akapit. \cite{goodfellow2016deep} %Nowy wymuszony akapit 

%%% PRZYKłŁADOWA TABELA %%%%%%%%%%%%%%%%%
                                        %
\begin{table}[ht]                       %
\centering                              %
\caption{Podpis nad tabelą,             %
         czcionka 10 pkt                %
         tak jak chcą w wymogach }      %
\begin{tabular}[t]{|c|c|c|c|}           %   
\hline                                  %
kolumna1 & kolumna2 & kolumna3 \\       %  
\hline                                  %       PRZYKłŁADOWA TABELA
komórka1 & komórka2 & komórka3 \\       %
komórka4 & komórka5 & komórka6 \\       %
komórka7 & komórka8 & komórka9 \\       %
\hline                                  %
\end{tabular}                           %
\label{tab:caption}                     %
\end{table}                             %
                                        %
%%% KONIEC PRZYKŁADOWEJ TABELI %%%%%%%%%%

\lipsum[2]   %Wygenerowanie przykładowego tekstu [USUŃ]

%%% PRZYKŁADOWY OBRAZEK %%%%%%%%%%%%%%%%%%%%%%%%%
                                                %
\begin{figure}[h!]{}                            %
    \centering                                  %
    \includegraphics[scale=0.4]{pobrane.png}    %       PRZYKŁADOWY OBRAZEK
    \caption{Podpis pod rysunkiem}              %
    \label{fig:logo}                            %
\end{figure}                                    %
                                                %
%KONIEC PZYŁADOWEGO OBRAZKA %%%%%%%%%%%%%%%%%%%%%


\lipsum[2] \cite{min2017deep} %Wygenerowanie przykładowego tekstu [USUŃ]

%%% PRZYKŁADOWE RÓWNANIE %%%%%%%%%%%%%%%%
                                        %
\begin{equ}[!ht]                        %
  \begin{equation}                      %
    E=mc^2                              %       PRZYKŁADOWE RÓWNANIE
  \end{equation}                        %
\end{equ}                               %
                                        %
%%% KONIEC PRZYKŁADOWEGO RÓWNANIA %%%%%%%

\lipsum[2] %Wygenerowanie przykładowego tekstu [USUŃ]

%%% PRZYKŁADOWY LISTING %%%%%%%%%%%%%%%%%%%%%%%%%%%%%%%%%%%%%
                                                             
\begin{lstlisting}[caption=Podpis listingu,label=code1]     
    #include <stdio.h>                                      
    int main () {                                           
        printf("Hello, World!");                            
        return 0;                                           
    }                                                       
\end{lstlisting}                                            
                                                            
%%% KONIEC PRZYKŁADOWEGO LISTINGU %%%%%%%%%%%%%%%%%%%%%%%%%%%          

\subsection{Podrozdział}

\lipsum[1-15]    %Wygenerowanie przykładowego tekstu [USUŃ]

\chapterstyle{default}
\subsubsection{Pod-podrozdział}

\lipsum[1-5]    %Wygenerowanie przykładowego tekstu [USUŃ]
