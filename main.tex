%Czcionka 12pkt
\documentclass[12pt]{report}

%Język Polski + Time New Romans
\usepackage[utf8]{inputenc}
\usepackage[T1]{fontenc}
\usepackage{tgtermes}
\usepackage{polski}

%automatyczne akapity przy nowym rozdziale
\usepackage{indentfirst} 

% Do tabel
\usepackage{multirow} 

 % ------------ Do Listingów (ustawianie języka, colorków itp) ------------ %
\usepackage{listings} 
\renewcommand{\lstlistlistingname}{Spis listingów}
\lstset {
    language=C,
    frame=single,
    tabsize=4,
    showstringspaces=false,
    numbers=left,
    upquote=true,
    commentstyle=\color{commentgreen},
    keywordstyle=\color{eminence},
    stringstyle=\color{red},
    basicstyle=\small\ttfamily,
    emph={int,char,double,float,unsigned,void,bool,return},
    emphstyle={\color{blue}},
    escapechar=\&,
    classoffset=1,
    keywordstyle=\color{blue},
    morekeywords={>,<,.,;,,,-,!,=,~},
    classoffset=0,
}
% ------------------------------------------------------------------------ %

% Do wgrania strony tytułowej jako pdfa
\usepackage{pdfpages}

%Do bibliografii
\usepackage[style=numeric,firstinits,maxbibnames=100,language=polish,sortlocale=pl_PL]{biblatex}
\addbibresource{biblio.bib}

%Marginesy + margines pod oprawę pod druk dwustronny :)
\usepackage[top=2.5cm,bottom=2.5cm,left=3cm,right=2.5cm,twoside]{geometry}
\usepackage{lipsum}

%Rozmiar tytułów rozdziałów, podrozdziałów i pod-podrozdziałów
\usepackage{sectsty}
\sectionfont{\fontsize{14}{13}\selectfont}
\subsectionfont{\fontsize{13}{12}\selectfont}
\subsubsectionfont{\fontsize {12}{11}\selectfont}

%Ustawienia podpisów tabel, rysunków, równań i listingów
\renewcommand{\thefigure}{\arabic{section}.\arabic{figure}}
\renewcommand{\thetable}{\arabic{section}.\arabic{table}}
\renewcommand{\theequation}{\arabic{section}.\arabic{equation}}
\usepackage{caption}
\DeclareCaptionFormat{myformat}{\fontsize{10}{10}\selectfont#1#2#3}
\captionsetup{format=myformat}
\usepackage{graphicx}
\DeclareCaptionType{equ}[][]

%Interlinia pojedyncza i długość akapitów
\renewcommand{\baselinestretch}{1}
\setlength{\parindent}{0.7cm}

%Możliwość dodanie obu streszczeń (ang i pol) na jednej stronie
\renewenvironment{abstract}{
    \vspace{150}
    \begin{center}%
        \bfseries\abstractname
    \end{center}}%

%## POCZAĘTEK DOKUMENTU ##%
\begin{document}

%#############################     STRONA TYTUŁOWA     ################################%
\begin{titlepage}       
\includepdf{pd_mgr_pl}              % Polecam wygenerować se stronę tytułową ze schematu ze strony w docxie przez google doca np. a potem wstawić jako plik PDF, bo inaczej to dużo roboty :)
\end{titlepage}                      
%######################################################################################%
\newpage
\shipout\null       % Pusta strona, aby drukowało się ładnie :) Można wywalić przy generowaniu wersji cyfrowej pracy

%############################       STRESZCZENIE    ###################################%
\begin{abstract}
    Streszczenie po polsku.         %Twoje streszczenie po polsku
\end{abstract}

\renewcommand{\abstractname}{Abstract} %Tak wiem, głupia metoda na zmianę nagłówka streszczenia, ale używając dedykowanego pakietu "babel" były problemu z resztą dokumentu więc zostawiam to tak.
\begin{abstract}
    Abstract in English.            % Twoje streszczenie po angielsku
\end{abstract}
%######################################################################################%
\newpage
\shipout\null       % Pusta strona, aby drukowało się ładnie :) Można wywalić przy generowaniu wersji cyfrowej pracy

\tableofcontents    % Spis treści
\newpage
\shipout\null       % Pusta strona, aby drukowało się ładnie :) Można wywalić przy generowaniu wersji cyfrowej pracy

%###############################   TREŚĆ GŁÓWNA PRACY ##################################%

\chapter{Nazwa rozdziału}

\section{Rozdział}
Zażółć gęślą jaźń :). Polskie znaki działają podczas kopiowania pdfa czyli działają też w systemie anty-plagiatowym. Pierwszy wiersz automatycznie wstawia akapit. \\ %nowa linia
\indent Zobacz do kodu jeżeli chcesz wstawić dodatkowy akapit. \cite{goodfellow2016deep} %Nowy wymuszony akapit 

%%% PRZYKłŁADOWA TABELA %%%%%%%%%%%%%%%%%
                                        %
\begin{table}[ht]                       %
\centering                              %
\caption{Podpis nad tabelą,             %
         czcionka 10 pkt                %
         tak jak chcą w wymogach }      %
\begin{tabular}[t]{|c|c|c|c|}           %   
\hline                                  %
kolumna1 & kolumna2 & kolumna3 \\       %  
\hline                                  %       PRZYKłŁADOWA TABELA
komórka1 & komórka2 & komórka3 \\       %
komórka4 & komórka5 & komórka6 \\       %
komórka7 & komórka8 & komórka9 \\       %
\hline                                  %
\end{tabular}                           %
\label{tab:caption}                     %
\end{table}                             %
                                        %
%%% KONIEC PRZYKŁADOWEJ TABELI %%%%%%%%%%

\lipsum[2]   %Wygenerowanie przykładowego tekstu [USUŃ]

%%% PRZYKŁADOWY OBRAZEK %%%%%%%%%%%%%%%%%%%%%%%%%
                                                %
\begin{figure}[h!]{}                            %
    \centering                                  %
    \includegraphics[scale=0.4]{pobrane.png}    %       PRZYKŁADOWY OBRAZEK
    \caption{Podpis pod rysunkiem}              %
    \label{fig:logo}                            %
\end{figure}                                    %
                                                %
%KONIEC PZYŁADOWEGO OBRAZKA %%%%%%%%%%%%%%%%%%%%%


\lipsum[2] \cite{min2017deep} %Wygenerowanie przykładowego tekstu [USUŃ]

%%% PRZYKŁADOWE RÓWNANIE %%%%%%%%%%%%%%%%
                                        %
\begin{equ}[!ht]                        %
  \begin{equation}                      %
    E=mc^2                              %       PRZYKŁADOWE RÓWNANIE
  \end{equation}                        %
\end{equ}                               %
                                        %
%%% KONIEC PRZYKŁADOWEGO RÓWNANIA %%%%%%%

\lipsum[2] %Wygenerowanie przykładowego tekstu [USUŃ]

%%% PRZYKŁADOWY LISTING %%%%%%%%%%%%%%%%%%%%%%%%%%%%%%%%%%%%%
                                                             
\begin{lstlisting}[caption=Podpis listingu,label=code1]     
    #include <stdio.h>                                      
    int main () {                                           
        printf("Hello, World!");                            
        return 0;                                           
    }                                                       
\end{lstlisting}                                            
                                                            
%%% KONIEC PRZYKŁADOWEGO LISTINGU %%%%%%%%%%%%%%%%%%%%%%%%%%%          

\subsection{Podrozdział}

\lipsum[1-15]    %Wygenerowanie przykładowego tekstu [USUŃ]

\chapterstyle{default}
\subsubsection{Pod-podrozdział}

\lipsum[1-5]    %Wygenerowanie przykładowego tekstu [USUŃ]
 % ZOBACZ PLIK CHAPTER1.TEX 

% W pliku "chapter.tex" jest cała treść pracy. Zamieściłem tam przykłady tabel, rysunków oraz równań :) % 

%Bibliografia - z pliku biblio.bib
\printbibliography

\listoftables   %Spis tabel
\listoffigures  %Spis rysunków
\lstlistoflistings %Spis listingów - nie wiem czy jest konieczny ale dodałem

\end{document}
